%% To get pdf-file:    (acroread below allows you to look at pdf-file)
%====================
% (A) if your figures are ps-files or eps-files  (and not pdf,png,jpeg,etc.):
%---------------------------------------------------------------------
% latex trafficflowtalk.tex ;trafficflowtalk.tex ; dvips trafficflowtalk.dvi -o ; ps2pdf trafficflowtalk.ps ; acroread trafficflowtalk.pdf
% 
% (B) if your figures are pdf,png,jpeg etc (and not ps-files or eps-files)
%---------------------------------------------------------------------
% pdflatex beamer_example.tex 
%
\documentclass[t]{beamer}
% Ben's option:
\usecolortheme{whale}       
% Sophie's option:
%\usepackage{beamerthemeBerkeley}
%
\usefonttheme[onlymath]{serif}   
\usepackage{graphicx}
\usepackage{caption}

\usepackage{subfig}
\usepackage{commath}

\setbeamertemplate{navigation symbols}{}  % turns off annoying nav. symbols

\title{Real Time Signal Processing with Symmetric and Asymmetric Support Intervals}
\author{Kevin Joyce and Lia Harrington \\
       University of Montana}
\date{May 14, 2015}
 
% for a recurring outline
% (this defines, what happens whenever you use \section{...}, namely
%     slide is created with title "Universality ..." and table of contents
%     which includes all section titles and highlights current section)
\AtBeginSection[]  
{
\begin{frame}
\frametitle{Outline} 
\tableofcontents[currentsection] 
\end{frame}  
}

\begin{document}

%%%%%%%%%%%%%%%%%%%%%%%%%%%%%%%%%%%%%%%%%%%%%%%%%%%%%%%%%%%%%%
\begin{frame}
\titlepage
\end{frame}
%%%%%%%%%%%%%%%%%%%%%%%%%%%%%%%%%%%%%%%%%%%%%%%%%%%%%%%%%%%%%
\section{Introduction and Motivation}
%%%%%%%%%%%%%%%%%%%%%%%%%%%%%%%%%%%%%%%%%%%%%%%%%%%%%%%%%%%%%
\begin{frame}
\frametitle{Importance of Real Time Signal Processing}
\vspace{.3cm}
\structure{What is real time signal processing?}
\vspace{.3cm}
\begin{itemize}
\item Applications
\begin{itemize}
\item Speech recognition
\item Audio signal processing
\item Video compression
\item Weather forecasting
\item Economic forecasting
\item Medical imagining (e.g., CAT, MRI)
\item And more...
\end{itemize}
\end{itemize}
%\begin{figure}
%\includegraphics[scale=.4]{women_stem.png}
%\caption{STEM Education Data and Trends NSF}
%\end{figure} 
\end{frame}
%%%%%%%%%%%%%%%%%%%%%%%%%%%%%%%%%%%%%%%%%%%%%%%%%%%%%%%%%%%%
\section{Problem Outline}
%%%%%%%%%%%%%%%%%%%%%%%%%%%%%%%%%%%%%%%%%%%%%%%%%%%%%%%%%%%%
\begin{frame}
\frametitle{What is the problem?}
\structure{Goal:} We wish to reconstruct some generated signal $\hat{x}$ that has been distorted by some error and convolution processes. \newline \vspace{.5cm}

\structure{Solution:} Take the convolution inverse of $\hat{x}$ to reconstruct the signal. 
\end{frame}
%%%%%%%%%%%%%%%%%%%%%%%%%%%%%%%%%%%%%%%%%%%%%%%%%%%%%%%%%%%%
\section{Problem Approach and Steps} 
%%%%%%%%%%%%%%%%%%%%%%%%%%%%%%%%%%%%%%%%%%%%%%%%%%%%%%%%%%%%
\begin{frame}
\frametitle{Problem Simulation}
\structure{Specify the main ingredients of simulated measurement system:}
\begin{itemize}
\item Finitely supported point spread function (influence function), 
  \begin{itemize}
    \item Symmetric case: $a_i = \frac 1{10}$ for $|i| < 15$.  
    \item Asymmetric case: $a_i = \frac 2{10} e^{-i/40}$ for $0\le i\le 40$.
  \end{itemize}
\item The covariance function $\phi$ for the signal $x$ is given by 
$$
\phi = \mathrm{Cov}\,(x) = b*b^{*}
$$
  where $b_i = \frac {21}{100}(1-|i|)$ for $|i| \le 7$. 
\item Measurement noise is modeled with a zero mean Gaussian $\nu$ with a specified $\sigma^2 = \frac 1{100}$.
\item Finally, the data is given by
$$
  y = a * x + \nu
$$
\end{itemize}
\end{frame}

\begin{frame}
  \frametitle{Reconstruction Operator $R$}  
  \begin{itemize}
    \item Following Lecture 13, we seek a reconstruction operator $R$ that is given by convolution with $r$ supported on a specified interval $\Delta$, so that $\widehat x = r*x$.
    \item Further, it was shown that
    $$
      H(r) = E( \widehat x - x )^2 = \langle P (r - P^{-1}q), r- P^{-1}q \rangle_{\Delta} + f_0 - \langle q, P^{-1}q\rangle_{\Delta}
    $$
    where $P$ is the operator associated with convolution by $p = a*\phi*a^* + \sigma^2\delta$ and $q = a*\phi$.
    \item So, for a given $\Delta$, the reconstruction kernel is uniquely determined by \alert{$r = P^{-1}q$}, and 
    $$
      \mathrm{Var}\, \widehat x = H_{min} = \alert{f_0 - \langle q, r\rangle_{\Delta}}.
    $$
  \end{itemize}
\end{frame}
%%%%%%%%%%%%%%%%%%%%%%%%%%%%%%%%%%%%%%%%%%%%%%%%%%%%%%%%%%%%
\section{Code Comments}
%%%%%%%%%%%%%%%%%%%%%%%%%%%%%%%%%%%%%%%%%%%%%%%%%%%%%%%%%%%%
\section{Results and Conclusions}
%%%%%%%%%%%%%%%%%%%%%%%%%%%%%%%%%%%%%%%%%%%%%%%%%%%%%%%%%%%%
\begin{frame}
\frametitle{References}

[1] Golubtsov, P. (2015). Theoretical Big Data Analytics course notes.

\end{frame}

\end{document}




\end{document}
